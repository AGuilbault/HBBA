\documentclass[12pt]{article}
\usepackage[utf8]{inputenc}
\usepackage{amsmath}
\usepackage[cm]{fullpage}

\parindent 0pt
\DeclareMathOperator*{\argmax}{arg\,max}

\begin{document}

\author{François Ferland}
\title{IW Translator - Notation Reference}

\maketitle

\section*{Introduction}

This document is an attempt to clear up some issues about the notation used in the IW Translator.
The most correct notation could be found in an unpublished paper (for IROS 2012), but still had some readability issues.
Furthermore, it didn't match the notation used in the old implementation of the IW Solver.
So, in preparation of a rewrite of the IW Solver into the new iw\_translator package, this document is meant to be used as a reference for both the implementation and future writing on the subject.

\section*{Solving process}

\subsection*{Indices}

\begin{itemize}
	\item A desire $d$,
	\item A strategy $i$,
	\item A resource $j$,
	\item A desire class $k$.
\end{itemize}

\subsection*{Static vectors and matrices}

These are specified by strategies and the system's configuration:

\begin{itemize}
	\item $u_{ik}$: Utility produced by strategy $i$ for class $k$,
	\item $c_{ij}$: Cost of activating strategy $i$ for resource $j$,
	\item $r_{ik}$: Minimum utility required by the activation of strategy $i$ for class $k$,
	\item $m_{j}$: Maximum cost allowed for resource $j$.
\end{itemize}

\subsection*{Dynamic vectors and matrices} 

These are specified by the current desires set:

\begin{itemize}
	\item $b_{dk}$: Minimum utility required by desire $d$ for class $k$,
	\item $v_{dk}$: Parameters of desire $d$ for class $k$,
	\item $s_{dk}$: Intensity of desire $d$ for class $k$,
\end{itemize}

\subsection*{Desire selection process}

Based on $b_{dk}$, $v_{dk}$ and $s_{dk}$, we aim to produce:

\begin{itemize}
	\item $g_{k}$: Minimum utility required selected for class $k$,
	\item $p_{k}$: Parameters selected for class $k$.
\end{itemize}

The technique is simple: for each class $k$, find the the most intense desire $d$:

\begin{itemize}
	\item $g_{k} = \{ b_{dk} | \argmax_d s_{dk} \} $,
	\item $p_{k} = \{ v_{dk} | \argmax_d s_{dk} \} $. 
\end{itemize}

\subsection*{Solving process}
The goal is to determine $a_{i}$, the activation state of strategy $i$, and
$f_{k}$, the consideration of desire class $k$, by respecting these constraints:

\begin{itemize}
	\item $\sum\limits_i a_i (u_{ik} - r_{ik}) \geq f_k g_k$, for each class $k$
        where $g_k > 0$, 
	\item $\sum\limits_i a_i u_{ik} = 0$ and $\sum\limits_i a_i r_{ik} = 0$, for each class $k$
        where $g_k = 0$, 
	\item $\sum\limits_i a_i c_{ij} \leq m_j$, for each resource $j$.
\end{itemize}

Where $a_i, f_k \in [0,1]$.

To find the best solution, the solving process maximizes $f \cdot s$, which 
favors consideration of the most intense desires in case the resource 
constraints cannot be met for all desire classes.
Optionally, the solving process maximizes $f \cdot s + p$, where $p = \sum\limits_k\sum\limits_i a_i u_{ik}$ for all $k$ where $g_k \neq 0$.
\end{document}

